\documentclass[12pt]{article}

\usepackage{amsthm}
\usepackage{graphicx}
\usepackage{float}

\newtheorem{theorem}{Theorem}[section]
\newtheorem{lemma}[theorem]{Lemma}
\newtheorem*{remark}{Remark}

\begin{document}
\section{Definitions}
    $G = (V, E)$ is a unoriented graph. $\omega(G)$ is the clique number, $\Delta(G)$ the maximum degree of $G$. 
    We will denote for $v \in V$ by $G-v$ the induced subgraph of $G$ that contains all the elements of $V$ but $v$. \\
    Let $C:=\{C_1, C_2, ... , C_k\}$ be a set of such that $\forall i \le k, |C_i|$ is $K_1, K_2$ or $K_3$. The triangle number of $G$
    is the minimal size of $C$. We denote it by $\Omega(G)$.

\section{Lemmas}
\begin{lemma}
    For any graph $G$, $\Omega(G) \le |A|$. 
\end{lemma}

\begin{remark}
    Proof is trivial; there is equality if $\omega(G) = 2$, so for forests, in particular.  
\end{remark}

\begin{lemma}
    Let $V_1$, $V_2$ be a partition of $V$ such that for $v_1 \in V_1$, $v_2 \in V_2$, 
    there exist no path between $v_1$ and $v_2$. Then $\Omega(V) = \Omega (V_1) + \Omega(V_2).$ 
\end{lemma}

\begin{remark}
    Proof is probably less easy but should not be too complicated. \\
    Discussing this lemma allows us to only think about connected graphs.
\end{remark}

\begin{theorem}[Triangle number of complete graphs]
    $$\Omega(K_n) = {n-1 \choose 2} \frac{n}{3}$$
\end{theorem}

\begin{proof}
    It suffice to count how many triangles there are in $G = K_n$.\\
    Consider $v\in V$. Notice that $d(v) = n-1$. As $G = K_n$, when we choose any two edges that have an end in $v$,
    the other two ends are neighbours themselves. Thus, we have exactly ${n-1 \choose 2}$ triangles that contains $v$ in $K_n$.\\
    By doing that for every vertex of $G$, and dividing by $3$ as we counted each triangle $3$ times, we obtain the result. 
\end{proof}

The following lemmas are corollaries:
\begin{lemma}
    $$\Omega(G) \le {|V|-1 \choose 2} \frac{|V|}{3}$$
\end{lemma}

\begin{lemma}
    $$\Omega(G) \ge {\omega(G)-1 \choose 2} \frac{\omega(G)}{3}$$
\end{lemma}



\end{document}