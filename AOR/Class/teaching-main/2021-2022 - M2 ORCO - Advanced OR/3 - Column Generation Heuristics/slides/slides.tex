\documentclass[10pt]{beamer}

\usepackage[utf8]{inputenc}
\usepackage[T1]{fontenc}
\usepackage{amsmath}
\usepackage{amsthm}
\usepackage{dsfont}
\usepackage{graphicx}
\usepackage{interval}
\usepackage{color}
\usepackage{xcolor,colortbl}
\usepackage[style=authoryear, maxcitenames=11, mincitenames=11, natbib=true, uniquename=false, uniquelist=false]{biblatex}
\addbibresource{main.bib}
\usepackage{multirow}
\usepackage{booktabs}
\usepackage{tikz}
\usetikzlibrary{patterns}
\usetikzlibrary{positioning}
\usetikzlibrary{decorations.pathreplacing,angles,quotes}
\usepgflibrary{arrows.meta,shapes.geometric}
\usepackage{algorithm}
\usepackage{algorithmicx}
\usepackage[noend]{algpseudocode}
\usepackage{dirtytalk}
\usepackage{hyperref}

\newcommand{\N}{\mathbf{N}}
\newcommand{\Z}{\mathbf{Z}}
\newcommand{\Q}{\mathbf{Q}}
\newcommand{\R}{\mathbf{R}}
\newcommand{\C}{\mathbf{C}}

\beamertemplatenavigationsymbolsempty%
\addtobeamertemplate{navigation symbols}{}{%
    \usebeamerfont{footline}%
    \usebeamercolor[fg]{footline}%
    \hspace{1em}%
    \insertframenumber/\inserttotalframenumber%
}

\author{Florian Fontan}
\title{Advanced Models and Methods in Operations Research \\ Column Generation Heuristics}
\date{November 23, 2021}

\begin{document}

\newcommand{\customcite}[1]{\citetitle{#1}, \citeauthor{#1}, \citeyear{#1}}

\AtBeginSection[]
{%
  \begin{frame}<beamer>
    \frametitle{Table of contents}
    \tableofcontents[currentsection]
  \end{frame}
}

\maketitle

\section{Exponential MILP formulations}

\begin{frame}
  \frametitle{Cutting Stock Problem, Description}

  Input:
  \begin{itemize}
    \item a capacity \alert{$C$}
    \item $n$ item types; for each item type $j = 1, \dots, n$, a weight \alert{$w_j$} and a demand \alert{$q_j$}
  \end{itemize}

  Problem:
  \begin{itemize}
    \item Pack all items such that the total weight of the items in a bin does not exceed the capacity.
  \end{itemize}

  Objective:
  \begin{itemize}
    \item Minimize the number of bin used.
  \end{itemize}
\end{frame}

\begin{frame}
  \frametitle{Cutting Stock Problem, Formulation}

  Let us define the $K$ possible patterns such that $x_j^k = q$ iff pattern $k$, $k = 1 \dots K$ contains $q$ copies of item type $j$

  \begin{itemize}
    \item Variables:
      \begin{itemize}
        \item $y^k \in \N$,
          $\forall k = 1 \dots K$.

          $y^k = q$ iff $q$ copies of pattern $k$ are used
      \end{itemize}

    \item Objective:
      \begin{displaymath}
        \min \sum_{k = 1}^{K} y^k
      \end{displaymath}

    \item Constraints:
      \begin{align*}
        \sum_{k = 1}^{K} x_j^k y^k = q_j && \forall j = 1 \dots n \
      \end{align*}
  \end{itemize}

  \pause Why is this formulation good compared to the classical one?
  \begin{itemize}
    \item \pause No big-M constraint
    \item Better relaxation
    \item Easier to write
  \end{itemize}
\end{frame}

%\mystar{y}{r}{d-r}{t-r}{p}{nom}
\def\mystar#1#2#3#4#5#6#7{
  % 1: y
  % 2: r_s
  % 3: d_s - r_s
  % 4: t_s - r_s
  % 5: p_s
  % 6: nom
  % 7: opacity
  \begin{scope}[shift={(#2,#1)},opacity=#7,transparency group]
    \draw[line width=1pt,color=blue,arrows={Bracket-Bracket}] (0.4,0) -- +(#3,0);
    \draw[line width=5pt,color=red] (0.4,0) ++(#4,0) -- ++(#5,0);
    \def\tmp{#6}\if\tmp\empty\else
    \node[star, star point height=3mm, minimum size=5mm, color=yellow, fill=yellow, text=black,scale=.4] at (0,0) {#6};
    \fi
  \end{scope}
}

\def\mystaronly#1#2#3#4{
  % 1: y
  % 2: r_s
  % 3: nom
  % 4: opacity
  \begin{scope}[shift={(#2,#1)},opacity=#4,transparency group]
    \node[star, star point height=3mm, minimum size=5mm, color=yellow, fill=yellow, text=black,scale=.4] at (0,0) {#3};
  \end{scope}
}

\begin{frame}
  \frametitle{Star Observation Scheduling Problem, Description}

  Input: a set \alert{$\cal M$} of nights and a set \alert{$\cal N$} of stars; for each star $j \in\cal N$, a scientific interest \alert{$w_j$}, an observation duration \alert{$p_j^i$} and a visibility window \alert{$\interval[open right]{r_j^i}{d_j^i}$}, depending on the night $i$ of the observation.
  
  \vfill
  
  \only<1>{%
    \begin{tikzpicture}
      % \mystar{y}{r}{d-r}{t-r}{p}{nom}
      \mystaronly{2.5}{-.2}{$1$}{1}
      \mystaronly{2.0}{-.2}{$2$}{1}
      \mystaronly{1.5}{-.2}{$3$}{1}
      \mystaronly{1.0}{-.2}{$4$}{1}
      \mystaronly{0.5}{-.2}{$5$}{1}
      \mystaronly{0.0}{-.2}{$6$}{1}
      \mystar{2.5}{0}{2.5}{0.25}{2}{}{1}     \mystar{2.5}{5.3}{3.0}{0.5}{2}{}{1}
      \mystar{2.0}{3.0}{1.5}{.25}{1}{}{1}      \mystar{2.0}{7.5}{1.5}{.25}{1}{}{1}
      \mystar{1.5}{1.5}{3.0}{0.5}{2}{}{1}      \mystar{1.5}{5.8}{2.8}{0.4}{2}{}{1}
      \mystar{1.0}{0.5}{2.0}{0.45}{1.1}{}{1}   \mystar{1.0}{5.2}{2.0}{0.45}{1.1}{}{1}
      \mystar{0.5}{3.5}{1.25}{0.25}{0.75}{}{1} \mystar{0.5}{8.2}{1.25}{0.25}{0.75}{}{1}
      \mystar{0}{2}{2.5}{.5}{1.5}{}{1}    \mystar{0}{6.7}{2.75}{.625}{1.5}{}{1}

      \draw[color=gray,dotted,line width=3pt] (.2,3)--(.2,-1) (5.4,3)--(5.4,-1) (10.1,3)--(10.1,-1);

      \node [gray] at (2.5,-.8) {nuit 1};
      \node [gray] at (8.3,-.8) {nuit 2};
    \end{tikzpicture}
  }
  \only<2>{%
    \begin{tikzpicture}
      % \mystar{y}{r}{d-r}{t-r}{p}{nom}
      \mystaronly{2.5}{-.2}{$1$}{1}
      \mystaronly{2.0}{-.2}{$2$}{.2}
      \mystaronly{1.5}{-.2}{$3$}{1}
      \mystaronly{1.0}{-.2}{$4$}{1}
      \mystaronly{0.5}{-.2}{$5$}{1}
      \mystaronly{0.0}{-.2}{$6$}{1}
      \mystar{2.5}{0}{2.5}{0}{2}{}{1}        \mystar{2.5}{5.3}{3.0}{0.5}{2}{}{.2}
      \mystar{2.0}{3.0}{1.5}{.25}{1}{}{.2}     \mystar{2.0}{7.5}{1.5}{.25}{1}{}{.2}
      \mystar{1.5}{1.5}{3.0}{0.5}{2}{}{1}      \mystar{1.5}{5.8}{2.8}{0.4}{2}{}{.2}
      \mystar{1.0}{0.5}{2.0}{0.45}{1.1}{}{.2}  \mystar{1.0}{5.2}{2.0}{0}{1.1}{}{1}
      \mystar{0.5}{3.5}{1.25}{0.5}{0.75}{}{1}  \mystar{0.5}{8.2}{1.25}{0.25}{0.75}{}{.2}
      \mystar{0}{2}{2.5}{.5}{1.5}{}{.2}   \mystar{0}{6.7}{2.75}{0}{1.5}{}{1}

      \draw[color=gray,dotted,line width=3pt] (.2,3)--(.2,-1) (5.4,3)--(5.4,-1) (10.1,3)--(10.1,-1);

      \node [gray] at (2.5,-.8) {nuit 1};
      \node [gray] at (8.3,-.8) {nuit 2};
    \end{tikzpicture}
  }
\end{frame}

\begin{frame}
  \frametitle{Star Observation Scheduling Problem, Formulation}

  For each night $i$, $i = 1 \dots m$, let us define the $K_i$ possible schedules such that $x_{i, j}^k = 1$ iff schedule $k$, $k = 1 \dots K_i$ of night $i$ contains star $j$

  \begin{itemize}
    \item Variables:
      \begin{itemize}
        \item $y_i^k \in \{0, 1\}$,
          $\forall i = 1 \dots m$, $\forall k = 1 \dots K_i$.

          $y_i^k = 1$ iff scheduled $k$ of night $i$ is selected
      \end{itemize}

    \item Objective:
      \begin{displaymath}
        \max \sum_{i = 1}^m \sum_{k = 1}^{K_i} \sum_{j = 1}^n w_j x_{i, j}^k y_i^k
      \end{displaymath}

    \item Constraints:
      \begin{align*}
        1 \le \sum_{k = 1}^{K_i} y_i^k \le 1 && \forall i = 1 \dots m \\
        \sum_{i = 1}^n \sum_{k = 1}^{K_i} x_{i, j}^k y_i^k = 1 && \forall j = 1 \dots n \\
      \end{align*}
  \end{itemize}
\end{frame}

\begin{frame}
  \frametitle{2D Guillotine Variable-sized Bin Packing, Description}

  Input:
  \begin{itemize}
    \item $n$ item types; for each item type $j = 1, \dots, n$, a width \alert{$w_j$}, a height \alert{$h_j$} and a demand \alert{$q_j$}
    \item $m$ bin types; for each bin type $i = 1, \dots, m$, a lower bound \alert{$l_i$}, an upper bound \alert{$u_i$} and a cost \alert{$c_i$}
  \end{itemize}

  Problem:
  \begin{itemize}
    \item Find a subset of guillotine patterns such that all item type demands and bin type use bounds are satisfied
  \end{itemize}

  Objective:
  \begin{itemize}
    \item Minimize the cost of the selected bins.
  \end{itemize}
\end{frame}

\begin{frame}
  \frametitle{2D Guillotine Variable-sized Bin Packing, Formulation}

  For each bin type $i$, $i = 1 \dots m$, let us define the $K_i$ possible patterns such that $x_{i, j}^k = q$ iff pattern $k$, $k = 1 \dots K_i$ of bin type $i$ contains $q$ copies of item type $j$

  \begin{itemize}
    \item Variables:
      \begin{itemize}
        \item $y_i^k \in \N$,
          $\forall i = 1 \dots m$, $\forall k = 1 \dots K_i$.

          $y_i^k = q$ iff $q$ copies of pattern $k$ of bin type $i$ are used
      \end{itemize}

    \item Objective:
      \begin{displaymath}
        \min \sum_{i = 1}^m \sum_{k = 1}^{K_i} c_i y_i^k
      \end{displaymath}

    \item Constraints:
      \begin{align*}
        l_i \le \sum_{k = 1}^{K_i} y_i^k \le u_i && \forall i = 1 \dots m \\
        \sum_{i = 1}^n \sum_{k = 1}^{K_i} x_{i, j}^k y_i^k = q_j && \forall j = 1 \dots n \\
      \end{align*}
  \end{itemize}
\end{frame}

\begin{frame}
  \frametitle{Other examples}

  Usually, variables represent:
  \begin{itemize}
    \item A bin/knapsack (for packing problems)
    \item The schedule of a machine (for parallel scheduling problems)
    \item The route of a vehicle (for vehicle routing problems)
    \item \dots
  \end{itemize}
\end{frame}

\section{Solving the relaxation by Column Generation}

\begin{frame}
  \frametitle{Introduction}
  
  \begin{itemize}
    \item With these formulations, generating all the variables is generally not possible since their number grows exponentially with the size of the problem.
    \item First we focus on solving the \alert{linear relaxation}
  \end{itemize}
\end{frame}

\begin{frame}
  \frametitle{The Column Generation procedure}
  
  \begin{itemize}
    \item We use the \alert{simplex algorithm}.
      \begin{itemize}
        \item At each iteration, it adds a variable of negative reduced cost to the current basis
        \begin{itemize}
          \item Objective:
            \begin{displaymath}
              \min \sum_{j = 1}^n c_j x_j
            \end{displaymath}
          \item Constraints:
          \begin{align*}
            \sum_{j = 1}^n a_{i, j} x_j \le b_j && \forall j = 1 \dots n \\
          \end{align*}
          \item Reduced cost of variable $x_j$:
            \begin{displaymath}
              c_j - a_{i, j} v_j
            \end{displaymath}
            with $v_j$ the dual value of constraint $j$.
        \end{itemize}
        \item It stops when there are no variable of negative reduced cost
      \end{itemize}
    \item The difference with the traditional simplex algorithm, is that here, it is not possible to loop through all the variables to find a variable of negative reduced cost, since they have not been all generated.
  \end{itemize}
\end{frame}

\begin{frame}
  \frametitle{The Column Generation procedure}
  \begin{itemize}
    \item Instead, finding a variable of negative reduced cost becomes of optimization problem

    \item Example with the Cutting Stock Problem
      \begin{itemize}
        \item Objective:
          \begin{displaymath}
            \min \sum_{k = 1}^{K} y^k
          \end{displaymath}

        \item Constraints:
          \begin{align*}
            \sum_{k = 1}^{K} x_j^k y^k = q_j && \forall j = 1 \dots n \\
          \end{align*}
      \end{itemize}

    \item Reduced cost of $y^k$:
      \begin{displaymath}
        1 - \sum_{j = 1}^n x_j^k v_j
      \end{displaymath}
      with $v_j$ the dual value of constraint $j$.
  \end{itemize}
\end{frame}

\begin{frame}
  \frametitle{The Column Generation procedure}
  
  \begin{itemize}
    \item We look for a variable $y^k$ such that
      \begin{displaymath}
        1 - \sum_{j = 1}^n x_j^k v_j < 0
      \end{displaymath}
    
    \item Finding a variable of negative reduced cost is equivalent to finding a pattern with total profit $\ge 1$ with the profit of item type $j$ being equal to $v_j$.

    \item In practice, we solve the problem as an optimization problem: we find the best solution of the Knapsack Problem and check if the reduced cost of the corresponding variable is negative.
  \end{itemize}

\end{frame}

\begin{frame}
  \frametitle{The Column Generation procedure}
  
  Summary:
  \begin{algorithmic}
    \Function{ColumnGeneration}{$P$}
    \State $Y \gets \text{initial set of columns}$
    \While{True}
    \State Solve the Linear Program $P'$ with variables from $Y$
    \State Look for a variable of negative reduced cost (\alert{Pricing Problem})
    \If {there is one}
    \State Add it to $Y$
    \Else
    \State \Return the solution of $P'$
    \EndIf
    \EndWhile
    \EndFunction
  \end{algorithmic}
\end{frame}

\begin{frame}
  \frametitle{Initial set of columns}

  \begin{itemize}
    \item To get dual values, the LP needs to be feasible
    \item With $0$ variable, the LP might be infeasible
      \begin{itemize}
        \item Example: Cutting Stock Problem, demand constraints are not satisfied
      \end{itemize}
    \item Therefore, we need to find a way to get an initial set of columns such that the LP is feasible
      \begin{itemize}
        \item Find a feasible solution and add the corresponding columns
          \begin{itemize}
            \item Example: Cutting Stock, Best Fit algorithm
            \item Drawback: Problem specific, additional work for the implementation of the heuristic
            \item Advantage: if the solution is good, it might speed up the column generation proceudre
          \end{itemize}
        \item Find manually a set of columns that ensures the LP to be feasible
          \begin{itemize}
            \item Example: create $n$ columns with only one item
          \end{itemize}
        \item Generate a dummy column with very high cost for each problematic constraint
          \begin{itemize}
            \item Advantage: not problem specific
            \item Drawback: numerical issue is the cost of the dummy columns is not well calibrated
          \end{itemize}
      \end{itemize}
  \end{itemize}
\end{frame}

\begin{frame}
  \frametitle{Star Observation Scheduling Problem}

  \begin{itemize}
    \item Objective:
      \begin{displaymath}
        \max \sum_{i = 1}^m \sum_{k = 1}^{K_i} \sum_{j = 1}^n w_j x_{i, j}^k y_i^k
      \end{displaymath}

    \item Constraints:
      \begin{align*}
        1 \le \sum_{k = 1}^{K_i} y_i^k \le 1 && \forall i = 1 \dots m && \alert{\text{dual } u_j} \\
        \sum_{i = 1}^n \sum_{k = 1}^{K_i} x_{i, j}^k y_i^k = 1 && \forall j = 1 \dots n && \alert{\text{dual } v_j} \\
      \end{align*}

    \item \pause Reduced cost of $y_i^k$:
      \begin{displaymath}
        \sum_{j = 1}^n w_j x_{i, j}^k - u_i - \sum_{j = 1}^n x_{i, j}^k v_j
        = \sum_{j = 1}^n (w_j - v_j) x_{i, j}^k - u_i
      \end{displaymath}

    \item \pause Finding a variable of maximum reduced cost reduces to solving $m$ Single Night Star Observation Scheduling Problems with targets with profit $w_j - v_j$.
  \end{itemize}
\end{frame}

\begin{frame}
  \frametitle{2D Guillotine Variable-sized Bin Packing, Pricing}
  
  \begin{itemize}
    \item Objective:
      \begin{displaymath}
        \min \sum_{i = 1}^m \sum_{k = 1}^{K_i} c_i y_i^k
      \end{displaymath}

    \item Constraints:
      \begin{align*}
        l_i \le \sum_{k = 1}^{K_i} y_i^k \le u_i && \forall i = 1 \dots m && \alert{\text{dual } u_j} \\
        \sum_{i = 1}^n \sum_{k = 1}^{K_i} x_{i, j}^k y_i^k = q_j && \forall j = 1 \dots n && \alert{\text{dual } v_j} \\
      \end{align*}

    \item \pause Reduced cost of $y_i^k$:
      \begin{displaymath}
        c_i - u_i - \sum_{j = 1}^n x_j^k v_j
      \end{displaymath}

    \item \pause Finding a variable of minium reduced cost reduces to solving $m$ 2D Guillotine Knapsack Problems with items with profit $v_j$ for each bin type.
  \end{itemize}
\end{frame}

\begin{frame}
  \frametitle{Transition}

  \begin{itemize}
    \item The Column Generation procedure solves the relaxation of the exponential formulation
    \item \pause Thus, it provides a valid bound
    \item \pause But it generally does not provide a feasible solution
    \item \pause How to exploit the Column Generation to get feasible solutions?
  \end{itemize}
\end{frame}

\section{Finding solutions after the Column Generation}

\begin{frame}
  \frametitle{The Branch-and-Price algorithm}
  
  \begin{itemize}
    \item LP-based branch-and-bound, the relaxation is solved by the Column Generation procedure in each node
    \item How to branch?
      \begin{itemize}
        \item Branching on columns of the exponential formulation? No, the pricing problem becomes to difficult
          \item Branching on the variables of the primal formulation?
            \begin{itemize}
              \item Bin Packing: branch on whether item $j$ is packed in bin $i$ or not. If yes, then the available bins have now different capacities and a Knapsack Problem for each capacity needs to be computed
            \end{itemize}
          \item Best solution for the Bin Packing: branch on whether two items are packed in the same bin or not. If yes, then they are merged into a single item. If no, then the subproblem becomes a Knapsack Problem with Conflicts which is strongly NP-hard
      \end{itemize}
      \item $\implies$ Complex to implement.
      \item It can be combined with cuts (Branch-and-Price-and-Cut). The added cuts might also change the pricing problem $\implies$ even more complex to implement
      \item Only exact method based on Column Generation, state-of-the-art exact method for many Vehicle Routing and Parallel Machine Scheduling Problems
  \end{itemize}
\end{frame}

\begin{frame}
  \frametitle{Solving the restricted master}

  \begin{itemize}
    \item The Column Generation procedure is executed once
    \item Solve the exponential formulation with a MILP solver using only the columns generated during the Column Generation procedure
    \item No guarantee to find the optimal solution (or even a feasible solution)
    \item Solving the MILP is computationaly expensive if many columns have been generated. It can take some time before finding a first solution
    \item It requires a good MILP solver 
  \end{itemize}
\end{frame}

\begin{frame}
  \frametitle{Heuristic Tree Search}

  Branching scheme:
  \begin{itemize}
    \item Root node: no column has been fixed
    \item Children: solve the relaxation by column generation, select the variable $y$ with the most integral value $v \neq 0$, for each possible value $v'$ of $y$ create one child.
    \item The discrepancy of a child is computed as:
      \begin{displaymath}
        \mathrm{disc}_\text{child} = \mathrm{disc}_\text{father} + | v' - v |
      \end{displaymath}
  \end{itemize}

  Algorithms:
  \begin{itemize}
    \item Greedy
    \item Limited Discrepancy Search
  \end{itemize}

  Note that the depth of the tree is of the order of the number of columns in a solution.
\end{frame}

\begin{frame}
  \frametitle{Additional tricks}

  \begin{itemize}
    \item Using a fast heuristic algorithm to solve the pricing problem. If the heuristic doesn't find a column of negative reduced cost:
      \begin{itemize}
        \item Case 1: Try with a more expensive exact algorithm
        \item Case 2: Stop the column generation procedure. The bound is not valid, therefore, it is not possible to use an exact Branch-and-price in this case. But the heuristics still work.
      \end{itemize}
    \item Generating columns without the simplex algorithm
      \begin{itemize}
        \item It might be faster than the column generation procedure
        \item It might be difficult to generate columns that fit well together
        \item No bound
        \item Then solve the restricted master or use a Heuristic Tree Search algorithm
      \end{itemize}
    \item Solving the restricted master with a heuristic algorithm
      \begin{itemize}
        \item Often, the master problem is a set covering or set packing problem for which heuristic algorithms have already been developed
        \item It might be faster than a MILP solver
      \end{itemize}
  \end{itemize}
\end{frame}

\section{\texttt{columngenerationsolverpy}}

\begin{frame}
  \frametitle{\texttt{columngenerationsolverpy}}

  \begin{itemize}
    \item A package that simplifies the implementation of Column Generation based algorithms
    \item Written in \texttt{Python3} (original version in \texttt{C++})
    \item \url{https://github.com/fontanf/columngenerationsolverpy}
    \item Install with: \texttt{pip3 install columngenerationsolverpy}
    \item It includes:
      \begin{itemize}
        \item The Column Generation algorithm
        \item The Greedy algorithm
        \item The Limited Discrepancy Search algorithm
      \end{itemize}
    \item To solve a problem, one needs to provide the exponential formulation and the solver for the Pricing Problem (able to take as input the currently fixed columns)
    \item The implementation of the Greedy algorithm and the Limited Discpreancy Search algorithm relies on the \texttt{treesearchsolverpy} package
  \end{itemize}
\end{frame}

\section{Conclusion}

\begin{frame}
  \frametitle{Conclusion}

  \begin{itemize}
    \item Column Generation: solving the relaxation of exponential formulations by generating the columns dynamically
    \item It can be embedded in a classical Branch-and-bound
      \begin{itemize}
        \item State-of-the art exact method for many Vehicle Routing and Parallel Machine Scheduling Problems
        \item Cumbersome to implement
      \end{itemize}
    \item It can be embedded in a Heuristic Tree Search framework
      \begin{itemize}
        \item Also state of the art heuristics for several problems
        \item Easier to implement
      \end{itemize}
    \item Works better when the number of elements in columns is small ($\le 20$)
  \end{itemize}
\end{frame}

\maketitle

\end{document}

